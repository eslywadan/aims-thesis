\documentclass[12pt,a4paper]{article}
\usepackage[utf8]{inputenc}
\usepackage[T1]{fontenc}
\usepackage{amsmath,amsfonts,amssymb}
\usepackage{geometry}
\usepackage{setspace}
\usepackage{titlesec}
\usepackage{enumitem}
\usepackage{parskip}

% Page setup
\geometry{margin=1in}
\doublespacing

% Section formatting
\titleformat{\section}{\large\bfseries}{\thesection}{1em}{}
\titleformat{\subsection}{\normalsize\bfseries}{\thesubsection}{1em}{}

\title{Hybrid Multi-Objective Optimization Framework Under Uncertainty for Integrated Business Planning}
\author{Esly Wadan Chou}
\date{2025-W34}

\begin{document}

\maketitle

\noindent\textbf{Note:} Going to discuss with Professor Chou

\section{Background, Problem, and Contributions (Summary)}

Global manufacturing industries, such as TFT-LCD production, face increasing challenges in integrated business planning (IBP) due to uncertainty, market volatility, and the need to balance competing objectives. Traditional methods often prioritize cost minimization, overlooking service levels, resilience, and sustainability. While AI and optimization methods hold promise, they remain fragmented and difficult to scale in industrial contexts.

This research proposes a \textbf{hybrid multi-objective optimization framework under uncertainty}, combining evolutionary optimization, Bayesian approaches, deep learning, and robust stochastic programming with AI-enabled forecasting and constraint programming. The framework aims to balance cost, quality, throughput, and resilience under uncertainty.

\textbf{Expected contributions:}

\begin{itemize}[leftmargin=*]
    \item \textbf{Theoretical:} Advance decision sciences by integrating stochastic, robust, and learning-based optimization into IBP.
    
    \item \textbf{Methodological:} Develop and benchmark scalable hybrid approaches combining optimization, machine learning, and metaheuristics for uncertainty-aware decision-making.
    
    \item \textbf{Practical:} Deliver an industry-oriented framework for TFT-LCD manufacturing with actionable insights, improved responsiveness, and best-practice guidelines for firms.
\end{itemize}

In summary, this study bridges the gap between cutting-edge optimization research and large-scale industrial implementation, offering both academic and practical value.

\section{Background and Motivation (Detailed)}

The rapid evolution of global manufacturing, particularly in high-tech industries such as TFT-LCD production, presents increasing challenges for integrated business planning (IBP). Traditional approaches to planning and optimization often struggle with uncertainties arising from market fluctuations, supply chain disruptions, and production variability. At the same time, firms are under pressure to balance multiple objectives, such as cost efficiency, lead-time reduction, service level improvements, and sustainability goals.

AI-enabled decision support systems have emerged as a promising avenue to address these complexities. Advances in optimization, machine learning, and data-driven forecasting offer opportunities to create planning systems that are not only adaptive and resilient but also capable of balancing multiple, often conflicting, objectives. However, the integration of these approaches into large-scale industrial settings remains a significant challenge and requires rigorous methodological and practical contributions.

\section{Research Problem Statement (Detailed)}

Despite advances in operations research and AI-driven optimization, existing IBP frameworks often face the following limitations:

\begin{itemize}[leftmargin=*]
    \item \textbf{Limited handling of uncertainty:} Deterministic methods fail to capture demand and supply variability in dynamic manufacturing environments.
    
    \item \textbf{Single-objective bias:} Many systems prioritize cost minimization while neglecting multidimensional decision needs such as service levels and sustainability.
    
    \item \textbf{Lack of scalability and adaptability:} State-of-the-art methods may work in controlled settings but struggle to scale in complex industrial environments.
    
    \item \textbf{Fragmented methodologies:} Research often focuses on isolated techniques (e.g., stochastic programming, evolutionary optimization, or machine learning) rather than an integrated framework.
\end{itemize}

This study addresses the gap by proposing a \textbf{hybrid multi-objective optimization framework under uncertainty} that integrates AI-enabled techniques with advanced operations research methods. The TFT-LCD industry serves as the focal application, where decision-making requires balancing multiple objectives while managing uncertainty at both strategic and operational levels.

\section{Expected Research Contributions (Detailed)}

This research is expected to make contributions in three major dimensions:

\subsection{Theoretical Advancement}

\begin{itemize}[leftmargin=*]
    \item Develop a hybrid framework that integrates multi-objective optimization under uncertainty with AI-enabled IBP.
    
    \item Extend decision sciences by formalizing methods to address uncertainty in manufacturing planning through stochastic, robust, and learning-based models.
    
    \item Contribute to the literature on evolutionary multi-objective optimization (EMOO), Bayesian optimization (BMOO), deep model-based optimization learning (DMOL), and robust stochastic programming (RSP).
\end{itemize}

\subsection{Methodological Innovation}

\begin{itemize}[leftmargin=*]
    \item Design and implement a scalable methodology combining optimization with machine learning and AI-driven forecasting.
    
    \item Benchmark hybrid approaches (e.g., EMOO, BMOO, DMOL, RSP) through systematic experiments.
    
    \item Show how constraint programming and metaheuristics can be combined with statistical learning to improve decision-making under uncertainty.
\end{itemize}

\subsection{Practical and Industrial Impact}

\begin{itemize}[leftmargin=*]
    \item Deliver an industry-oriented framework for TFT-LCD manufacturing that enhances responsiveness, resilience, and efficiency.
    
    \item Provide actionable insights into how AI-enabled optimization reduces risks and balances competing objectives.
    
    \item Develop guidelines and best practices for firms seeking to integrate advanced optimization methods into planning systems.
\end{itemize}

\end{document}